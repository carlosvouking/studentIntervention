
\documentclass[12pt]{article}
\usepackage[english]{babel}
\usepackage[utf8x]{inputenc}
\usepackage{amsmath}
\usepackage{etoolbox}
\usepackage{changepage}
\usepackage{titlesec}
\usepackage[parfill]{parskip}
\usepackage[margin=1in]{geometry}
\usepackage{times}
\usepackage[numbers,super]{natbib}
\usepackage{enumitem} %get rid of spaces in listened

% For the images and graphics
\usepackage{subfig} % For subfigures in floats
\usepackage[section]{placeins}
\makeatletter
 \@ifpackageloaded{tex4ht}{%
\usepackage[dvips]{color,graphicx}
    \usepackage[tex4ht]{hyperref}
    }{%
      \usepackage[pdftex]{graphicx}
      \usepackage{hyperref}
          }
\makeatother
\graphicspath{ {/Users/omojumiller/mycode/MachineLearningNanoDegree/Machine-Learning-Project/studentIntervention/images/} } %Path to images


% For cutesy tables
\usepackage{multirow}
\usepackage[table]{xcolor}
\usepackage{longtable}
\usepackage{array}
\usepackage{booktabs} % To draw thick lines in table.
\usepackage{tablefootnote} % To allow footnotes in table.
\newcommand{\nextitem}{\par\hspace {\labelsep} \textendash \hspace {\labelsep}} % For list inside table cell

\newcommand{\HRule}{\rule{\linewidth}{0.5mm}}

%----------------------------------------------------------------------------------------
%  TITLE SECTION
%----------------------------------------------------------------------------------------
\title{\large \textbf{Building a Student Intervention System: An Udacity Nanodegree ML Project}} % using \large makes the title approximately 14 pt.
% Author info isn't included for the Annual Conference but some regional conferences might request it.
\author{Omoju Miller}
%\author{\normalsize Author Name\\
%\normalsize email@example.com\\
%\normalsize Name of Your Department\\\
%\normalsize Your Institution Name}
\date{\today} % This leaves the date blank.

\makeatletter % This gets the margins for the title set.
\patchcmd{\@maketitle}{\begin{center}}{\begin{adjustwidth}{0.5in}{0.5in}\begin{center}}{}{}
\patchcmd{\@maketitle}{\end{center}}{\end{center}\end{adjustwidth}}{}{}
\makeatother

\begin{document}
\raggedright
\maketitle
\thispagestyle{empty}
\pagestyle{empty}



%----------------------------------------------------------------------------------------
%  PAPER CONTENTS
%----------------------------------------------------------------------------------------
\section*{Introduction}




\section*{Models}
%-------------------D E C I S I O N  T R E E  C L A S S I F I E R ---------------------%

\subsection*{Decision Tree Classifier}
\begin{itemize} [noitemsep,nolistsep]
\item What is the theoretical $O(n)$ time \& space complexity in terms of input size?
\item What are the general applications of this model? What are its strengths and weaknesses?
\item Given what you know about the data so far, why did you choose this model to apply?
\end{itemize} 


\setlength{\extrarowheight}{1.5pt}
\begin{table}[!htbp]
\caption{Result of training with a DecisionTreeClassifier} %title of the table
\centering % centering table
\begin{tabular}{|p{6cm}|p{1.5cm}|p{1.5cm}|p{1.5cm}|} % creating four columns
\hline % inserts single-line
& \multicolumn{3}{c|}{Training set size}\\[5pt]
\cline{2-4} 
& 100 & 200 & 300\\[0.5ex]
\hline % inserts single-line

Training time (secs)   &       0.001 & 0.001 & 0.002 \\
Prediction time (secs)   &     0.000 & 0.000 & 0.000 \\
F1 score for training set  &   1.000 & 1.000 & 1.000 \\
F1 score for test set    &     0.683 & 0.703 & 0.758 \\
\hline % inserts single-line
\end{tabular}
\label{decisionTreeTable}
\end{table}

%-------------------S V M ---------------------%

\subsection*{Support Vector Machine}
\begin{itemize} [noitemsep,nolistsep]
\item What is the theoretical $O(n)$ time \& space complexity in terms of input size?
\item What are the general applications of this model? What are its strengths and weaknesses?
\item Given what you know about the data so far, why did you choose this model to apply?
\end{itemize} 


\setlength{\extrarowheight}{1.5pt}
\begin{table}[!htbp]
\caption{Result of training with a Support Vector Machine} %title of the table
\centering % centering table
\begin{tabular}{|p{6cm}|p{1.5cm}|p{1.5cm}|p{1.5cm}|} % creating four columns
\hline % inserts single-line
& \multicolumn{3}{c|}{Training set size}\\[5pt]
\cline{2-4} 
& 100 & 200 & 300\\[0.5ex]
\hline % inserts single-line

Training time (secs)   &       0.008 & 0.010 & 0.051 \\
Prediction time (secs)   &     0.000 & 0.000 & 0.000 \\
F1 score for training set  &   0.909 & 0.853 & 0.830 \\
F1 score for test set    &     0.767 & 0.769 & 0.779 \\
\hline % inserts single-line
\end{tabular}
\label{decisionTreeTable}
\end{table}




%----------------------------------------------------------------------------------------
\section*{Conclusion}

This paper has laid out some of the challenge of 













%----------------------------------------------------------------------------------------

\end{document}  